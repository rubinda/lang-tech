% To je predloga za poročila o domačih nalogah pri predmetih, katerih
% nosilec je Blaž Zupan. Seveda lahko tudi dodaš kakšen nov, zanimiv
% in uporaben element, ki ga v tej predlogi (še) ni. Več o LaTeX-u izveš na
% spletu, na primer na http://tobi.oetiker.ch/lshort/lshort.pdf.
%
% To predlogo lahko spremeniš v PDF dokument s pomočjo programa
% pdflatex, ki je del standardne instalacije LaTeX programov.

\documentclass[a4paper,11pt]{article}
\usepackage{a4wide}
\usepackage{fullpage}
\usepackage[utf8x]{inputenc}
\usepackage[slovene]{babel}
\selectlanguage{slovene}
\usepackage[toc,page]{appendix}
\usepackage[pdftex]{graphicx} % za slike
\usepackage{setspace}
\usepackage{color}
\definecolor{light-gray}{gray}{0.95}
\usepackage{listings} % za vključevanje kode
\usepackage{hyperref}
\renewcommand{\baselinestretch}{1.2} % za boljšo berljivost večji razmak
\renewcommand{\appendixpagename}{Priloge}

\lstset{ % nastavitve za izpis kode, sem lahko tudi kaj dodaš/spremeniš
language=Python,
basicstyle=\footnotesize,
basicstyle=\ttfamily\footnotesize\setstretch{1},
backgroundcolor=\color{light-gray},
}

\title{Risanje sinusoid}
\author{David Rubin (david.rubin@student.um.si)}
\date{\today}

\begin{document}

\maketitle

\section{Uvod}

Implementirajte program za popravljanje pravopisnih napak. Program naj vsebuje naslednje korake:
\begin{itemize}
    \item Ustvarite slovar, ki za vsako besedo vsebuje frekvenco pojavitev v učni množici. Vse besede so predstavljene z malimi črkami.
    \item Napišite funkcijo, ki za podano besedo vrne vse besede na razdalji urejanja 1. Operacije urejanja so brisanje, transpozicija, zamenjava in vstavljanje.
    \item Napišite funkcijo, ki za podano besedo vrne vse besede na razdalji urejanja 2. Pri tem uporabite funkcijo za vračanje besed, ki so na razdalji 1.
    \item Napišite funkcijo, ki za podano besedo vrne vse besede iz slovarja na razdalji urejanja 2.
    \item Napišite funkcijo, ki za podano besedo vrne seznam kandidatk in njihovih verjetnosti urejanja. Uporabimo poenostavljen model, ki uporablja naslednja pravila:
        \begin{itemize}
        \item Če je podana beseda že v slovarju, je ta beseda prva in zadnja v seznamu kandidatk. Njena verjetnost urejanja je 1.
        \item Če v slovarju ni podana besede, v seznam kandidatk dodamo besede, ki so v slovarju na razdalji urejanja 1. Njihove verjetnosti urejanja se normalizirajo glede na frekvence v slovarju in vsoto vseh frekvenc kandidatk.
        \item Če v slovarju ni podane besede in kandidatk na razdalji urejanja 1, v  seznam kandidatk dodamo besede, ki so v slovarju na razdalji urejanja 2. Njihove verjetnosti urejanja se normalizirajo glede na frekvence v slovarju in vsoto vseh frekvenc kandidatk.
        \item Če ni izpolnjen nobeden od predhodnih pogojev, je kandidatka podana beseda z verjetnostjo urejanja 1.
		\end{itemize}    
    \item Napišite funkcijo, ki bo z uporabo jezikovnega modela, ki ste ga implementirali na eni od prejšnjih vaj, popravila podano poved s pomočjo naslednjih pravil:
        \begin{itemize}
        \item Izračunamo verjetnost povedi brez uporabe kandidatk.
        \item Izračunamo verjetnost povedi, pri kateri uporabimo le eno od kandidatk. Ko računamo verjetnost n-grama, verjetnost množimo z verjetnostjo urejanja kandidatke.
        \item Za kandidatke, ki niso v slovarju uporabimo verjetnost 1/V.
        \item Rezultat je poved, ki ima maksimalno verjetnost. Tukaj si pomagamo z logaritmi.
		\end{itemize}
\end{itemize}

\section{Poročilo}

Rešitev sem implementiral kot Python program (za delovanje potrebuje kneser\_ney iz prejšnje naloge), ki je priložen poročilu (\textit{spellcheck.py}). V sklopu ocenjevanja, sem uporabil 100 povedi iz priloženega \textit{holbrook.dat}, kjer se pojavi 1 napaka. Po zagonu programa se generirajo datoteke, ki vsebujejo dejanske povedi (razbrane iz podanega korpusa), povedi popravljenje s pomočjo Kneser-Ney glajenja in povedi, kjer se za besedo kjer je definirana napaka uporabi tista z največjo verjetnostjo urejanja. Rezultati so sledeči:

\begin{center}
  \begin{tabular}{ | l | c | r |}
    \hline
    Metoda 				& Uspešnost 	& Št. zadetih besed \\ \hline
    s KN 				& 0.267		& 27/101 \\ \hline
    verjetnost urejanja 	& 0.198		& 20/101 \\ \hline
 
  \end{tabular}
\end{center}
V tabeli rezultatov je KN metoda, kjer verjetnost povedi ocenjujemo s pomočjo Kneser-Ney, pri verjetnosti urejanja pa le določimo, katera beseda je najverjetnejša za podano napako. Uspešnost meri koliko napak je uspešno popravila metoda, Število razlik pa pove, koliko besed se loči od pravilne povedi (torej lahko se zgodi, da je pri Kneser-Ney ocenila, da so še druge besede pri pravilno zapisanih verjetnejše).

\end{document}
